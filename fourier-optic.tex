\documentclass{article}
\usepackage{ctex}
\usepackage{mathrsfs}
\usepackage{amsthm,amsmath,amssymb}
\usepackage[T1]{fontenc}

\usepackage[marginal]{footmisc}
\usepackage[OT2,T1]{fontenc}
\DeclareSymbolFont{cyrletters}{OT2}{wncyr}{m}{n}
\DeclareMathSymbol{\Sha}{\mathalpha}{cyrletters}{"58}
\newcommand*{\dif}{\mathop{}\!\mathrm{d}}
\title{傅里叶光学}
\author{Kissshot-Acerolaorion-Heartunderblade}
\date{\today}
\begin{document}
\maketitle
\tableofcontents
本人不学无术,上课不认真听,到期中考试,临时抱佛脚,适逢学习\LaTeX,所
以用\LaTeX整理复习笔记。这里假设读者对傅里叶变换及其性质和一些特
殊函数有印象,本项目的其他文档有对这些的总结,读者可以先阅读本文档,有
疑问再去其他的文档查找。
\section{引言}
\paragraph{}
傅里叶光学,顾名思义,使用傅里叶变换
\footnote{
  傅里叶变换定义为:$F(\eta,\xi)=\mathscr{F}[f(x,y)]=\int\limits_{-\infty}^{+\infty}f(x,y)e^{-2\pi i(\eta x+\xi y)}\dif x\dif y$\\
  傅里叶逆变换是:
  $f(x,y)=\mathscr{F}^{-1}[F(\eta,\xi)]=\int\limits_{-\infty}^{+\infty}F(\eta,\xi)e^{2\pi
    i(\eta x+\xi y)}\dif \eta\dif\xi$
}
研究光学问题的一门学科。傅里叶光学
的一个数学基础是亥姆霍兹方程解的唯一性和线性性质。因为方程是线性的,所
以可以把边界条件分解成一种简单函数的叠加,分别研究每一种情况,最后利用线性性
质将不同情况的解进行叠加。这是一种将复杂问题分解成简单问题的一种方法,
这也是傅里叶变换的思想。而亥姆霍兹方程的唯一性条件保证我们可以猜解,像
电镜法一样,保证边界条件相同的情况下对亥姆霍兹方程给一个实际情况,利用
这一实际情况简化求解,这点十分重要,我们在下面章节中可以看见。
\section{角谱理论}
\subsection{角谱}
\paragraph{}
角谱就是将单色光场傅里叶变换的结果。在这里我们只研究单色光,
实际光场可以看成是单色光的叠加,用到的数学工具也是傅里叶变换,保证这一
过程有效的同样也是亥姆霍兹方程的线性和解的唯一性。众所周知,在光的传播中,遇
到的衍射孔径大于光波长的几倍在不太靠近孔径,光波的矢量性是不明显的,虽
然光的$\vec{E}$和$\vec{B}$是矢量,但是实验证明,这时候可以看成是一个复
数的标量形式。即:\[U_0(x,y,z)=A\exp{i\varphi}\] $U_0$代表改点的光的矢
量,$A$代表光的强度的平方根$\varpi$代表相位。而傅里叶变换是将这这一个
写成以下形式:
\[U_0=\int\limits_{-\infty}^{\infty}A(\eta,\xi,\zeta)e^{2\pi i(\eta
    x_0+\xi y_0+\zeta z_0)}\dif x_0 \dif
  y_0 \dif z_0\]我们研究光学问题,一个比较关心的问题在$z$给定情况下
$x$和$y$与光场的关系,在这一限制中,光的标量$U_0$可以由一个三自变量的函数
$U_0(x,y,z)$化为两个自变量$U_0(x,y)$,因为自变量$z$可以看成是一个条件,
我们关心的是在与$z$轴垂直的平面的光场分布。这时角谱可以简化为
$A(\eta,\xi)$。
\paragraph{}
$\eta$和$\xi$称为空间频率,它的物理意义马上就可以看到。首先将傅里叶变
换写成求和的形式,或者说是傅里叶级数的形式,在数学上这是由积分变成求和。
\begin{equation}
  \label{eq:1}
  U_0=\sum_{\eta \xi}A(\eta,\xi)e^{2\pi i(\eta x_0+\xi y_0)}
\end{equation}
$e^{2\pi i(\eta x_0+\xi y_0)}$代表了一束波矢为$(2\pi\eta,2\pi\xi)$的平
面波。对于这一点,可以通过对比平面波的公式得出。这种角谱将平面光场分解
为一系列平面波。对于空间中传播的波,可以分解为单色波,对于每个单色波来
说,存在$\zeta$满足条件\[\eta^2+\xi^2+\zeta^{2}=\frac{1}{\lambda^2}\],
即$2\pi(\eta,\xi,\zeta)$可以作为空间中的波长为$\lambda$的光波的波矢的在$x$、
$y$、$z$的三个方向的分解,这样的平面波公式自然是$e^{2\pi
  i(\eta,\xi,\zeta)\cdot(x_0,y_0,z_0)}$,在$xy$平面上的光场分布自然是回
归到了\ref{eq:1}的情形。
\subsection{角谱的传播}
众所周知,光在沿$z$轴传播过程中,$xy$平面上的光场分布会发生变化,所以
在传播过程中,角谱也会随$z$发生变化。在这里我们仅考虑单色波。光传播过
程满足亥姆霍兹方程$\nabla^2U+k^2U=0$(关于这一部分详见附录)。对于一
个给定的平面光场,无论他如何产生的,我们可以把他看作为一系列产生平面光
的光源产生的结果而不会引发任何问题,这一点类似于电镜法,电镜法将电场在
满足边界条件的情况下进行假想一个产生这一条件的电场源,由这个电场源求出
的电场不会有问题,这里也一样。由于亥姆霍兹方程的线性性质,我们可以将
$U_0$的每一束平面光分别研究,总光场分布的结果一定是这些单独结果的叠加,
单色光叠加成实际光源和实际光源分解同样是如此的。
\paragraph{}
对于单个平面波传播方程为:\[U=Ae^{2\pi i(\eta x+\xi y+\zeta z)}\]其中,
$A$为$\eta$、$\xi$、$z$的函数,代表的就是角谱,将这一方程带入亥姆霍兹方程可
以得到\[(\nabla^2+k^2)U=0\],展开得\[\frac{\dif^2A}{\dif
    z^2}+A(k^2-4\pi^{2}\eta^2-4\pi^2\xi^2)=0\]显然,方程的解为:
\begin{equation}
  \label{eq:2}
  A=Ce^{iz \sqrt{k^2-4\pi^2\eta^2-4\pi^2\xi^2}}
\end{equation}
$C$为和初值有关的常数,现在我们假设初始的光场角谱为$A_0$,则$C$可以简单
地取为$A_0$,$A_0$与只$\eta$、$\xi$有关,所以在传播$z$的距离后角谱变成
了
\begin{equation}
  \label{eq:3}
  A_z=A_0e^{iz \sqrt{k^2-4\pi^2\eta^2-4\pi^2\xi^2}}
\end{equation}
这就是角谱的传播规律。
\paragraph{}
由这也可以看出,当$k^2<4\pi^{2}(\eta^2+\xi^2)$时,开根会引入虚数,造成角谱传播过
程中的衰减,空间频率足够高的平面波将会消失掉。在这里做一个近似,将
$k^2<4\pi^2(\eta^2+\xi^2)$的地方角谱将近似为0,并定义:
\begin{equation}
  \label{eq:4}
  H(\eta,\xi)=\left \{
    \begin{array}{rcl}
      e^{iz \sqrt{k^2-4\pi^2\eta^2-4\pi^2\xi^2}}&&{k^2>4\pi^2(\eta^2+\xi^2)}\\
      0&&{}
    \end{array}
  \right
  .
\end{equation}
$H(\eta,\xi)$称为该系统的传递函数。
在旁轴近似下,即:
\[k^2>>\eta^2+\xi^2\]这时,将传递函数$H(\eta,\xi)$的指数项做近似
\[iz \sqrt{k^2-4\pi^2\eta^2-4\pi^2\xi^2}\Rightarrow izk
  \sqrt{1-4\pi^2\frac{\eta^2+\xi^2}{k^{2}}}\Rightarrow
  izk(1-2\pi^2\frac{\eta^2+\xi^2}{k^{2}})\]
最终得到传递函数可以近似为:
\begin{equation}
  \label{eq:5}
  H(\eta,\xi)=e^{izk}e^{-iz\pi\lambda(\eta^2+\xi^2)}
\end{equation}
\subsection{泰伯效应}
在傍轴近似下,如果入射光场为周期性光场,光场会出现随$z$变化的周期性的
重复,这一现象叫做泰伯效应。
将初始为周期性光场,将初始的光场进行傅里叶展开:
\[U_0=\sum_{-\infty}^{+\infty}c_{n,m}e^{2\pi i(\frac{nx}{d_1}+\frac{my}{d_2})} \]
求初始光强的角谱:
\[A_0=\mathscr{F}[\sum_{-\infty}^{+\infty}c_ne^{2\pi
    i(\frac{nx}{d_1}+\frac{my}{d_2})}]=\sum_{-\infty}^{+\infty}c_{n,m}\mathscr{F}[e^{2\pi
    i(\frac{nx}{d_1}+\frac{my}{d_2})}]\]
指数函数的傅里叶变换的结果是$\delta$函数
\footnote{
  $\delta(x)$是在零点为$+\infty$其余点都为零的一类泛函,并满足条件:
  $\int\limits_{-\infty}^{+\infty}\delta(x)\dif x=1$
}
,所以角谱可以写成:
\[A_0=\sum_{-\infty}^{+\infty}c_{n,m}\delta(\eta-\frac{n}{d_1},\xi-\frac{m}{d_2})\]
传播距离$z$之后,角谱变成:
\[A=A_0H=e^{ikz}\sum_{-\infty}^{+\infty}c_{n,m}e^{-i\pi\lambda
    z(\eta^2+\xi^2)}\delta(\eta-\frac{n}{d_1},\xi-\frac{m}{d_2})\]
\[\Rightarrow e^{ikz}\sum_{-\infty}^{+\infty}c_{n,m}e^{-i\pi\lambda
    z(\frac{n^2}{d_1^2}+\frac{m^2}{d_2^2})}\delta(\eta-\frac{n}{d_1},\xi-\frac{m}{d_2})\]

对于任意的$z$处角谱和$z+\frac{2d_1^2d_2^2}{\lambda}$处角谱只差一个常数
$e^{ik \frac{2d_1^2d_2^2}{\lambda}}$,这一点可以直接将这两个值带入验证,
这个常数是一个模为1的复数并不影响光场分布,只是对相位产生影响,所以入
射光场是周期函数,光场会随$z$而周期性发生变化。

\section{菲涅尔衍射}
上一章我们介绍了角谱理论,现在我们要将这一理论应用到菲涅尔衍射之中。
\subsection{菲涅尔衍射}

现在假设傍轴近似,将\ref{eq:5}带入到角谱的传播规律\ref{eq:3}
中\[A=A_0e^{izk}e^{-iz\pi\lambda(\eta^2+\xi^2)}\]傅里叶反变换可以得到空
间中的光场分布
\footnote{
  这里应用了傅里叶变换的性质
  $\mathscr{F}[fg]=\mathscr{F}[f]*\mathscr{F}[g]$\\
  其中,*代表卷积,卷积定义
  为:$f*g=\int\limits_{-\infty}^{+\infty}f(x_0,y_0)g(x-x_0,y-y_0)\dif
  x_0\dif y_0$
  \\关于这的详细证明过程可以参见本项目的其他文档}
\[U=\mathscr{F}^{-1}[A_0e^{izk}e^{-iz\pi\lambda(\eta^{2}+\xi^2)}]=\mathscr{F}^{-1}[A_0]*\mathscr{F}^{-1}[e^{ikz}e^{-iz\pi\lambda(\eta^2+\xi^2)}]\]
带入$A_0$定义可得:
\[\mathscr{F}^{-1}[A_0]=U_0\]
\[
  \mathscr{F}^{-1}[e^{ikz}e^{-iz\pi\lambda(\eta^2+\xi^2)}]=e^{ikz}\int\limits_{-\infty}^{+\infty}\int\limits_{-\infty}^{+\infty}e^{-iz\pi\lambda(\eta^2+\xi^2)}e^{2\pi
    i(\eta x+\xi y)}\dif\eta\dif\xi\]
\[\Rightarrow
  e^{ikz}\int\limits_{-\infty}^{+\infty}\int\limits_{-\infty}^{+\infty}e^{iz\pi\lambda((\eta-\frac{x}{z\lambda})^2+(\xi-\frac{y}{z\lambda})^2)}e^{i\pi
    \frac{x^2+y^2}{z\lambda}}\dif\eta\dif\xi\]
\[\Rightarrow e^{ikz}e^{i\pi
    \frac{x^2+y^2}{z\lambda}}\int\limits_{-\infty}^{+\infty}\int\limits_{-\infty}^{+\infty}e^{iz\pi\lambda((\eta-\frac{x}{z\lambda})^2+(\xi-\frac{y}{z\lambda})^2)}\dif\eta\dif\xi\]
\[\Rightarrow \frac{1}{iz\lambda}e^{ikz}e^{i\pi \frac{x^2+y^2}{z\lambda}}\]
将两式做卷积:
\[U(x,y)=\frac{1}{iz\lambda}e^{ikz}\int\limits_{-\infty}^{+\infty}U_0(x_0,y_0)e^{i\pi
    \frac{(x-x_0)^2+(y-y_0)^2}{z\lambda}}\dif x_0\dif y_0\]
\[\Rightarrow \frac{1}{iz\lambda}e^{ikz}e^{ik
    \frac{x^2+y^2}{2z}}\int\limits_{-\infty}^{+\infty}U_0(x_0,y_0)e^{ik
    \frac{x_0^2+y_0^2}{2z}}e^{-ik \frac{xx_0+yy_0}{z}}\dif x_0\dif
  y_0\]
\[\Rightarrow \frac{1}{iz\lambda}e^{ikz}e^{ik
    \frac{x^2+y^2}{2z}}\mathscr{F}[U_0e^{ik
    \frac{x_0^2+y_0^2}{2z}}]|_{\eta=\frac{x}{z\lambda},\xi=\frac{y}{z\lambda}}\]
于是菲涅尔衍射的公式应该为:
\begin{equation}
  \label{eq:6}
  U=\frac{1}{ikz}e^{izk}e^{ik \frac{x^2+y^2}{2z}}\mathscr{F}[U_0e^{ik \frac{x_0^2+y_0^2}{2z}}]|_{\eta=\frac{x}{z\lambda},\xi=\frac{y}{z\lambda}}
\end{equation}
\subsection{夫琅和费衍射}
当$z$足够的大且孔径足够的小的时候,也就是说,衍射屏可以近似足够远的时
候,满足条件:
\[z>>k(x_0^2+y_0^2)_{max}\]
这时候,光场分布可以简化为:
\[U=\frac{1}{iz\lambda}e^{ik
    \frac{x^2+y^2}{2z}}e^{ikz}\mathscr{F}[U_0]|_{\eta=\frac{x}{z\lambda},\xi=\frac{y}{z\lambda}}\]
光强分布为:
\begin{equation}
  \label{eq:7}
  I=U^{*}U=\frac{1}{z^2\lambda^2}|\mathscr{F}[U_0]|^2=\frac{1}{z^2\lambda^2}|A_0(\frac{x}{z\lambda},\frac{y}{z\lambda})|^2
\end{equation}
\subsection{衍射初始角谱}
衍射是光通过障碍物之后发生的现象,前面章节中讨论了衍射的光场分布,这一
节之中讨论衍射的初始角谱。
\paragraph{}
初始的角谱并不能够直接通过障碍物之前的光场求得,因为障碍物本身就会影响
角谱,例如平行于$z$轴的一个平面光照射到一个衍射的小孔上,平面光的角谱
是一个$\delta$函数,但是通过小孔之后的角谱一定不是一个$\delta$函数,因
为一个小孔的透光无论如何也是不能够由一个平面光产生的。
\paragraph{}
一般来说,障碍物对光场的影响可以写成以下形式:
\[U_i\tau=U_0 \]
$U_i$代表障碍物前的光场。上式定义了函数$\tau$,$\tau$表示障碍物对光强的
影响,是个复数,既有强度的影响又有相位的影响。一般来说,$\tau$与障碍物
的性质有关,障碍物对光的吸收对相位的影响反映到$\tau$上。将上式做傅里叶
变换:
\[A_i*T=A_0\]
$A_0$等于入射角谱和$T(\eta,\xi)$的卷积。将$T(\eta,\xi)$称为物体的频谱。
\subsection{巴比涅原理}
假设一个透过率为$\tau_0$的一个障碍物,将$\tau_0$分成两部分,$\tau_1$和
$\tau_2$,即:
\[\tau_0=\tau_1+\tau_2\]
有光场为$U_i$角谱为$A_i$的入射光场照射障碍物$\tau_0$,然后在经过衍射成
像光场变为:
\[U=U_0H=(\tau_0*A_0)H=(\tau_1*A_0)H+(\tau_2*A_0)H\]
即:
\[U=U_1+U_2\]
由此可以看出,在只有障碍物$\tau_1$和只有障碍物$\tau_2$的情况下的光场分
别为$U_1$和$U_2$,这两个
简单相加会是$\tau_1+\tau_2$的光场。如果$\tau_0$为没有障碍物,即$\tau_0=1$,则
$\tau_1$和$\tau_2$是互补屏,光场互补。这叫做巴比涅原理。
\section{衍射实例}
\subsection{单缝衍射}
如果是一束平行于$z$轴的平面光照射到一个矩形孔,这时
$A_i=\delta(\eta,\xi)$,而
$\tau=rect(\frac{x_0}{a})rect(\frac{y_0}{b})$,其中函数$rect$的定义如下:
\[
  rect(\frac{x}{a})=\left \{
    \begin{array}{rcl}
      1 && |\frac{x}{a}|\leq \frac{1}{2}\\
      0 && {}
    \end{array}
  \right
  .
\]
$\tau$进行傅里叶变换的到$T$
\[T(\eta,\xi)=\mathscr{F}[\tau]=\mathscr{F}[rect(\frac{x_0}{a})]\mathscr{F}[rect(\frac{y_0}{b})]=ab\mathop{}\!sinc(a\eta)sinc(b\xi)\]
$sinc$函数定义为:
\[sinc(x)=\frac{sin(\pi x)}{\pi x}\]
将函数$rect$带入傅里叶变化的公式容易得,$rect$的傅里叶变换的结果是
$sinc$函数。
\[A_0(\eta,\xi)=A_i*T=\int\limits_{-\infty}^{+\infty}ab\delta(\eta-\eta_0,\xi-\xi_0)sinc(a\eta_0)sinc(b\eta)\dif\eta_0\dif\xi_0\]
\[\Rightarrow T(\eta,\xi)\]
矩形孔径让光的空间频率展宽,也就是说,平行光经过孔径后发生了偏离原有传
播方向--原来是直线传播--的情况发生,这就是平常说的衍射。设初始光强为$I_0$考察矩形孔径的
夫琅和费衍射的光强分布为:
\[I=\frac{I_0}{z^2\lambda^2}|T(\frac{x}{z\lambda},\frac{y}{z\lambda})|^2\]
\[\Rightarrow I=\frac{a^2b^2I_0}{z^2\lambda^2}sinc^2(\frac{x}{z\lambda})sinc^2(\frac{y}{z\lambda})\]
当这个孔径是只有一维有限制的情况下上式可以简化为单缝衍射公式。
\subsection{双缝衍射}
继续考虑平行于$z$轴平面波的夫琅和费衍射,双缝可以看成两个单缝的情况,双缝衍射的透过率为:
\[\tau=rect(\frac{x-d}{a})rect(\frac{y}{b})+rect(\frac{x+d}{a})rect(\frac{y}{b})\]
其中,令:
\[\tau_1=rect(\frac{x-d}{a})rect(\frac{y}{b}),\tau_2=rect(\frac{x+d}{a})rect(\frac{y}{b})\]
$\tau_1$和$\tau_2$分别是一个有位移的单缝衍射,根据巴菲涅原理,总光场是
这两个单缝衍射的叠加。两个缝分别进行夫琅和费衍射的光场分布可以用
\ref{eq:7}加上傅里叶变换的位移定理
\footnote{
  位移定理:$\mathscr{F}[f(x-x_0,y-y_0)]=e^{-2\pi i(\eta x_0+\xi
    y_0)}F(\eta,\xi)$
  \begin{proof}
    \[\mathscr{F}[f(x-x_0,y-y_0)]=\int\limits_{-\infty}^{+\infty}f(x-x_0,y-y_0)e^{-2\pi
        i(\eta x+\xi y)}\dif x\dif y\]
    \[\Rightarrow e^{-2\pi i(\eta x_0+\xi
        y_0)}\int\limits_{-\infty}^{+\infty}f(x-x_0,y-y_0)e^{-2\pi i(\eta
        (x-x_0)+\xi(y-y_0))}\dif(x-x_0)\dif(y-y_0)\]
    \[\Rightarrow e^{-2\pi i(\eta x_0+\xi y_0)}F(\eta,\xi)\]
  \end{proof}

}
求得:
\[U_1=\frac{1}{iz\lambda}e^{ikz}e^{ik
    \frac{x^2+y^2}{2z}}T(\frac{x}{z\lambda},\frac{y}{z\lambda})e^{2\pi
    i\eta d}\]
\[U_2=\frac{1}{iz\lambda}e^{ikz}e^{ik
    \frac{x^2+y^2}{2z}}T(\frac{x}{z\lambda},\frac{y}{z\lambda})e^{-2\pi
    i\eta d}\]
其中$T$为单缝衍射的小孔的频率,总光场为:
\[U=U_1+U_2=\frac{2}{iz\lambda}e^{ikz}e^{ik
    \frac{x^2+y^2}{2z}}T\cos{2\pi i\eta d}\]
可以看出,由于总光强中有$\cos$函数的存在,光场会出现周期性的条纹,真实
光强会是单缝衍射和双缝干涉的叠加,这点和光学中的内容是一致的。

\subsection{圆孔衍射}
如果入射的孔径是一个圆孔,即:
\[
  \tau=cicr(\frac{r}{a})=\left \{
    \begin{array}{rcl}
      1&&x^2+y^2\leq a^2\\
      0&&{}
    \end{array}
  \right
  .
\]
首先求小孔的频率
\footnote{
  这里将坐标换成极坐标:
  \[x=r\cos{\theta},y=r\sin{\theta}\]
  \[\eta=\rho\cos{\phi},\xi=\rho\sin{\phi}\]}:
\[\mathscr{F}[\tau]=\int\limits_{-\infty}^{+\infty}cicr(\frac{r}{a})e^{2\pi
    ir\rho(\cos{\theta}\cos{\phi}+\sin{\theta}\sin{\phi})}r\dif
  r\dif\theta\]
这里,$cicr$是一个与$\theta$无关的一个函数,可以先对$\theta$进行积分:
\[\mathscr{F}[\tau]=\int\limits_{-\infty}^{+\infty}cicr(\frac{r}{a})r\dif
  r \int\limits_{0}^{2\pi}e^{-2\pi i
    r\rho(\cos{\theta}\cos{\phi}+\sin{\theta}\sin{\phi})}\dif\theta\]
积分后面一项是一类特殊函数贝塞尔函数:
\[J_0(2\pi r\rho)=\frac{1}{2\pi}\int\limits_{0}^{2\pi}e^{-2\pi i
    r\rho\cos(\theta-\phi)}\dif \theta\]
于是整个积分可以变为:
\[2\pi\int\limits_{0}^{+\infty} cicr(\frac{r}{a})J_0(2\pi
  r\rho)r\dif r\]
事实上,对于任意的与$\theta$无关的函数$g_r(r)$均可以照此思路化简:
\begin{equation}
  \label{eq:8}
  \mathscr{F}[g_r]=\mathscr{B}[g_r]=2\pi\int\limits_{0}^{+\infty}g_rJ_0(2\pi
  r\rho)r\dif r
\end{equation}
这个公式叫做傅里叶-贝塞尔变换,用符号$\mathscr{B}$表示。最终的变换结果与$\phi$无关。其逆变换为:
\begin{equation}
  \label{eq:9}
  \mathscr{B}^{-1}[G_\rho]=2\pi \int\limits_0^{+\infty}\rho
  G_{\rho}J_0(2\pi r\rho)\dif \rho
\end{equation}
圆域函数的傅里叶-贝塞尔变换为$\frac{aJ_1(2\pi a\rho)}{\rho}$,由此可以
求出圆孔的夫琅和费衍射光强为:
\[I=\frac{a^2J_1^2(2\pi a \frac{r}{z\lambda})}{zr\lambda}\]
可以看出,在随着$r$递增下,光强$I$递减的,函数$J_1$是一个有无穷多根
的函数,在零点到$J_1$第一个根的这一区域叫做爱里斑。这一区域内聚集着绝
大多数的光的能量,可以认为平行于$z$轴的光照射过来形成一个爱里斑。这个
斑的尺寸影响着光学仪器的分辨本领。
\paragraph{}
夫琅和费衍射代表着一种无穷远光源的一种衍射,像人的眼睛,一般的镜头都可
以用夫琅和费衍射来分析光学分辨本领。设两个点光源经过一个光学系统形成两
个爱里斑,这个光学系统镜头的直径为$D$,两个爱里斑之间如果足够小,这两个
爱里斑就可以视为重合,这两个点无法分辨,一般是将两个爱里斑如果是圆心距
离大于两个爱里斑的半径,这两个斑视为可分辨的。
\paragraph{}
$J_1$的第一个根的值约为3.83。所以:
\[\frac{r}{z}\approx 1.22 \frac{\lambda}{D}\]
旁轴近似下:
\[\theta\approx \frac{r}{z}\approx 1.22 \frac{\lambda}{D}\]
$\theta$为爱里斑中心对圆孔的夹角,这称为仪器的最小分辨角,象征着仪器的
分辨本领。
\paragraph{}
对于菲涅尔衍射,可以求出它在$z$轴上的光场分布。取零点$(0,0)$,这时带入
\ref{eq:6}可以得:
\[U=\frac{1}{ikz}e^{izk}\mathscr{F}[U_0e^{ik \frac{x_0^2+y_0^2}{2z}}]\]
其中
\[\mathscr{F}[U_0e^{ik
    \frac{x_0^2+y_0^2}{2z}}]=\int\limits_{-\infty}^{+\infty}circ(\frac{r}{a})e^{ik
    \frac{x_0^2+y_0^2}{2z}}\dif x_0\dif y_0\]
由于在零点处,傅里叶变换中的$e^{2\pi i(\eta x_0+\xi y_0)}$的一项值为1,
这时候有简单的积分形式。进一步推导可得:
\[\Rightarrow 2\pi\int\limits_0^ae^{ik \frac{r^2}{2z}}r\dif r\]
所以光场的表达式可以写成:
\[U=-e^{ikz}(e^{i \frac{ka^2}{2z\lambda}}-1)\]
光强为:
\[I=2-e^{i \frac{ka^2}{z\lambda}}-e^{i
    \frac{ka^2}{z\lambda}}=4\sin^2(\frac{\pi a^2}{2z\lambda})\]
这个结果也和直观上一致,当距离变远,光强会随着距离以$\frac{1}{z^2}$的
形式变化,而这正是能量随着距离散开的速度。
\subsection{光栅}
\subsubsection{理想光栅}
$\tau$具有周期性,则称这种结构为光栅。
\[\tau(x,y)=\tau(x+d_1,y+d_2)\]
将$\tau$可以写成傅里叶级数的形式:
\[\tau=\sum_{-\infty}^{+\infty}c_{n,m}e^{2\pi
    i(\frac{n}{d_1}x+\frac{m}{d_2}y)}\]
其中的系数为:
\[c_{n,m}=\frac{1}{d_1d_2}\int\limits_0^{d_1}\int\limits_0^{d_2}\tau(x,y)e^{-2\pi
    i(\frac{n}{d_1}x+\frac{m}{d_2}y)}\dif x\dif y\]
令函数$\tau_0$定义如下:
\[\tau_0=\tau rect(\frac{x}{d_1})rect(\frac{y}{d_2})\]
可以看出,这是将$\tau$的一个周期取出的结果。
\[d_1d_2c_{n,m}=\int\limits_{-\frac{d_1}{2}}^{\frac{d_1}{2}}\int\limits_{-\frac{d_2}{2}}^{\frac{d_2}{2}}\tau
  e^{-2\pi i(\frac{n}{d_1}x+\frac{m}{d_2}y)}\dif
  x\dif y=\int\limits_{-\infty}^{+\infty}\tau_0e^{-2\pi i(\eta x+\xi
    y)}\dif x\dif y|_{\eta=\frac{n}{d_1},\xi=\frac{m}{d_2}}\]
\[\Rightarrow
  c_{n,m}=\frac{1}{d_1d_2}\mathscr{F}[\tau_0]|_{\eta=\frac{n}{d_1},\xi=\frac{m}{d_2}}\]
求光栅的频谱:
\[T=\sum_{-\infty}^{+\infty}\mathscr{F}[\tau_0]|_{\eta=\frac{n}{d_1},\xi=\frac{m}{d_2}}\mathscr{F}[e^{2\pi
    i(\frac{n}{d_1}x+\frac{m}{d_2}y)}]\]
\begin{equation}
  \label{eq:10}
  T=\frac{1}{d_1d_2}\mathscr{F}[\tau_0]\sum_{-\infty}^{+\infty}\delta(x-\frac{n}{d_{1}},y-\frac{m}{d_2})
\end{equation}
一个光栅平行于$z$轴的平面光的在夫琅和费衍射下的光强可以写成:
\[I=\frac{1}{z^2\lambda^2}|\mathscr{F}[\tau_0]|^2|\sum_{-\infty}^{+\infty}\delta(\frac{x}{z\lambda}-\frac{n}{d_1},\frac{y}{z\lambda}-\frac{m}{d_2})|^2\]
$\delta$函数的平方可能并不能看出它的意义出来,但是这里我们可以把它将他
变为指数函数。现在令:
\[\Sha_{\frac{1}{d_1}\frac{1}{d_2}}(x,y)=\sum_{-\infty}^{+\infty}\delta(x-\frac{n}{d_1},y-\frac{m}{d_2})\]
$\Sha$为周期函数,将这一函数进行傅里叶展开,它的系数为:
\[c_{n,m}=d_1d_2
  \int\limits_{-\frac{1}{2d_1}}^{\frac{1}{2d_1}}\int\limits_{-\frac{1}{2d_2}}^{\frac{1}{2d_2}}\delta(x,y)e^{-2\pi
    i(\frac{n}{d_1}x+\frac{m}{d_2}y)}\dif
  x\dif y\]
这里我们进行了化简,因为在这一个区间上只有一个$\delta$函数是不恒为零的。
所以这时候啊可以得到:
\[\Sha_{\frac{1}{d_1}\frac{1}{d_2}}(x,y)=d_1d_2\sum_{-\infty}^{+\infty}e^{2\pi
    i(nd_1x+md_2y)}\]
这时候光强的最后一项意义已经很明显了,它代表所有这些周期性结构干涉的结
果,所以总光强为一个单位衍射和周期性结构干涉的叠加。后一项可能会出现极
大值,但是前一项的零点和后一项的极值重合的现象叫做衍射缺级。
如果$\tau_0=rect(\frac{x}{a})$的一维光栅就是我们熟悉的线光栅。带入上节的公
式可以得到它的光强的表达式为:
\[I=a^2sinc^2(a
  \frac{x}{z\lambda})|\Sha_{\frac{1}{d_1}\frac{1}{d_2}}(\frac{x}{z\lambda},\frac{y}{z\lambda})|^2\]
式中第一项表示的是$\tau_0$带来的影响,后一项是周期性结构带来的一项。在
理想光栅中,由周期型结构引起的是条纹宽度为零的一系列周期条纹。
\paragraph{}
如果透射率为余弦形式,即:
\[\tau=\frac{1}{2}+\frac{m}{2}\cos{2\pi \frac{x}{d}}\]
入射光不是平面波,而是任意函数$f(x,y)$的形式,透过光栅之后的广场为:
\[U=f\tau\]
写成角谱的形式为:
\[A=\mathscr{F}[f]*\mathscr{F}[\tau]\]
其中
\[\mathscr{F}[\tau_1]=\frac{1}{2}\delta(\eta)+\frac{m}{4}\delta(\eta-\frac{1}{d})+\frac{m}{4}\delta(\eta-\frac{1}{d})\]
所以最终的角谱应该为
\[A=\frac{F(\eta,\xi)}{2}+\frac{m}{4}F(\eta+\frac{1}{d},\xi)+\frac{m}{4}F(\eta-\frac{1}{d},\xi)\]
这表明余弦光栅对入射光进行了调制,对角谱进行了移动。
\subsubsection{实际光栅}
实际的光栅是做不到理想光栅那样的无限大,总是在一定范围内的光栅。
假设这一个范围是用函数$\tau_1$限制的,例如函数$rect$可以做这一种限制。
这时,光栅的频谱就改变了。
\[\tau'=\tau_1\tau\]
\[\Rightarrow
  T'=T_1*T=\frac{1}{d_1d_2}\sum_{-\infty}^{+\infty}T_1(\eta-\frac{n}{d_1},\xi-\frac{m}{d_2})T(\frac{n}{d_1},\frac{m}{d_2})\]
式中求和的每一项分别代表了不同位置的周期性结构的影响。实际光栅的光强为:
\[I=\frac{1}{z^2\lambda^2}\frac{1}{d_1d_2}(\sum_{-\infty}^{+\infty}T_1(\eta-\frac{n}{d_1},\xi-\frac{m}{d_2})T_0(\eta,\xi))^2\]
现在我们假设光栅为线光栅狭缝大小为$a$间距为$d$,限制的$\tau_1$取为:
\[\tau_1=rect(\frac{x_0}{L})rect(\frac{y_0}{L})\]
这时光栅频谱可以化成:
\[T'=\frac{aL^2}{d}\sum_{-\infty}^{+\infty}sinc(\frac{an}{d})sinc(L(\eta-\frac{n}{d}))sinc(L\xi)\]
假设光栅的间距足够的大,达到不同的狭缝的爱里斑不会重叠,这时可以近似的
认为狭缝的光全部集中到爱里斑内,不同狭缝的光不会重叠,这时候光强中的最
后一项对无穷级数求和的然后平方的这一项可以忽略交叉项,因为这个求和中的
不同项是代表不同的狭缝的影响,交叉项代表着两个狭缝相互影响,这是后这个
可以忽略。这时候的光强化为:
\[I=\frac{a^2L^4}{z^2d^2\lambda^2}\sum_{-\infty}^{+\infty}sinc^2(\frac{an}{d})sinc^2(L(\frac{x}{z\lambda}-\frac{n}{d}))sinc^2(L
  \frac{Ly}{z\lambda})\]
在$\frac{x}{z\lambda}-\frac{n}{d}$位置取最大值,并在傍轴近似条件下取得
$\sin{\theta}=\frac{x}{z}$得到光栅方程为:
\[d\sin{\theta}=n\lambda\]
如果光栅的透过率是以余弦形式变化的,光栅就变成了余弦光栅。
令:
\[\tau=\frac{1}{2}+\frac{m}{2}\cos(2\pi \frac{x}{d}),\tau_1=rect(\frac{x}{L})rect(\frac{y}{L})\]
这时
\[T=\mathscr{F}[\tau_1]=\frac{1}{2}\delta(\eta)+\frac{m}{4}\delta(\eta-\frac{1}{d})+\frac{m}{4}\delta(\eta-\frac{1}{d})\]
\[T_1=L^2sinc(L\eta)sinc(L\xi)\]
于是
\[T'=T*T_1=sinc(L\xi)(\frac{L^2}{2}sinc(L\eta)+\frac{L^2m}{4}sinc(L(\eta+\frac{1}{d}))+\frac{L^2m}{4}sinc(L(\eta-\frac{1}{d})))\]
继续假设光栅的空间频率足够的小,不同项之间的交叉项无意义,可以求出光强:
\[I=\frac{L^4}{4z^2\lambda^2}sinc(\frac{Ly}{z\lambda})(sinc^2(\frac{Lx}{z\lambda})+\frac{m}{4}sinc^2(\frac{L(x+\frac{\lambda
      z}{d})}{z\lambda})+\frac{m}{4}sinc^2(\frac{L(x-\frac{z\lambda}{d})}{z\lambda}))\]
\section{附录}
正文从角谱理论出发构建了整个的衍射理论,附录中将会用传统的方法推导出衍
射的公式。
\subsection{线性介质中光的传播}
首先,光在介质中传播遵从麦克斯韦方程组:
\begin{equation}
\label{eq:11}
\begin{array}{l}
  \nabla \cdot \vec{E}=\frac{\rho}{\varepsilon_0\varepsilon_{r}}\\
  \nabla \times \vec{E}=-\frac{\partial \vec{B}}{\partial t}\\
  \nabla \cdot \vec{B}=0\\
  \nabla \times \vec{B}=\mu_0\vec{\jmath}+\mu_0\varepsilon_0\varepsilon_r
  \frac{\partial \vec{E}}{\partial t}
\end{array}
\end{equation}
在线性介质中没有电流,$\vec{\jmath}$
\subsection{亥姆霍兹-基尔霍夫定理}

\end{document}
%%% Local Variables:
%%% mode: latex
%%% TeX-master: t
%%% End:
