\documentclass{article}
\usepackage{ctex}
\usepackage{mathrsfs}
\usepackage{amsthm,amsmath,amssymb}
\newcommand*{\dif}{\mathop{}\!\mathrm{d}}
\title{傅里叶光学}
\date{\today}
\begin{document}
\maketitle
\tableofcontents
本人不学无术,上课不认真听,到期中考试,临时抱佛脚,适逢学习\LaTeX,所
以用\LaTeX整理复习笔记。
\section{引言}
\paragraph{}
傅里叶光学,顾名思义,使用傅里叶变换研究光学问题的一门学科。傅里叶光学
的一个数学基础是亥姆霍兹方程解的唯一性和线性性质。因为方程是线性的,所
以可以把边界条件分解成一种简单函数的叠加,分别研究每一种情况,最后利用线性性
质将不同情况的解进行叠加。这是一种将复杂问题分解成简单问题的一种方法,
这也是傅里叶变换的思想。而亥姆霍兹方程的唯一性条件保证我们可以猜解,像
电镜法一样,保证边界条件相同的情况下对亥姆霍兹方程给一个实际情况,利用
这一实际情况简化求解,这点十分重要,我们在下面章节中可以看见。
\section{角谱理论}
\paragraph{角谱}就是将单色光场傅里叶变换的结果。在这里我们只研究单色光,
实际光场可以看成是单色光的叠加,用到的数学工具也是傅里叶变换,保证这一
过程有效的同样也是亥姆霍兹方程的线性和解的唯一性。众所周知,在光的传播中,遇
到的衍射孔径大于光波长的几倍在不太靠近孔径,光波的矢量性是不明显的,虽
然光的$\vec{E}$和$\vec{B}$是矢量,但是实验证明,这时候可以看成是一个复
数的标量形式。即:\[U_0(x,y,z)=A\exp{i\varphi}\] $U_0$代表改点的光的矢
量,$A$代表光的强度的平方根$\varpi$代表相位。而傅里叶变化是将这这一个
写成以下形式:
\[U_0=\int\limits_{-\infty}^{\infty}A(\eta,\xi,\zeta)e^{2\pi i(\eta
    x_0+\xi y_0+\zeta z_0)}\dif x_0 \dif
  y_0 \dif z_0\]我们研究光学问题,一个比较关心的问题在$z$给定情况下
$x$和$y$与光场的关系,在这一限制中,光的标量$U_0$可以由一个三自变量的函数
$U_0(x,y,z)$化为两个自变量$U_0(x,y)$,因为自变量$z$可以看成是一个条件,
我们关心的是在与$z$轴垂直的平面的光场分布。这时角谱可以简化为
$A(\eta,\xi)$。
\paragraph{}
$\eta$和$\xi$称为空间频率,它的物理意义马上就可以看到。首先将傅里叶变
换写成求和的形式,或者说是傅里叶级数的形式,在数学上这是由积分变成求和。
\begin{equation}
  \label{eq:1}
  U_0=\sum_{\eta \xi}A(\eta,\xi)e^{2\pi i(\eta x_0+\xi y_0)}
\end{equation}
$e^{2\pi i(\eta x_0+\xi y_0)}$代表了一束波矢为$(2\pi\eta,2\pi\xi)$的平
面波。对于这一点,可以通过对比平面波的公式得出。这种角谱将平面光场分解
为一系列平面波。对于空间中传播的波,可以分解为单色波,对于每个单色波来
说,存在$\zeta$满足条件\[\eta^2+\xi^2+\zeta^{2}=\frac{1}{\lambda^2}\],
即$2\pi(\eta,\xi,\zeta)$可以作为空间中的波长为$\lambda$的光波的波矢的在$x$、
$y$、$z$的三个方向的分解,这样的平面波公式自然是$e^{2\pi
  i(\eta,\xi,\zeta)\cdot(x_0,y_0,z_0)}$,在$xy$平面上的光场分布自然是回
归到了\ref{eq:1}的情形。
\subsection{角谱的传播}
众所周知,光在沿$z$轴传播过程中,$xy$平面上的光场分布会发生变化,所以
在传播过程中,角谱也会随$z$发生变化。在这里我们仅考虑单色波。光传播过
程满足亥姆霍兹方程$\nabla^2U+k^2U=0$(关于这一部分详见附录)。对于一
个给定的平面光场,无论他如何产生的,我们可以把他看作为一系列产生平面光
的光源产生的结果而不会引发任何问题,这一点类似于电镜法,电镜法将电场在
满足边界条件的情况下进行假想一个产生这一条件的电场源,由这个电场源求出
的电场不会有问题,这里也一样。由于亥姆霍兹方程的线性性质,我们可以将
$U_0$的每一束平面光分别研究,总光场分布的结果一定是这些单独结果的叠加,
单色光叠加成实际光源和实际光源分解同样是如此的。
\paragraph{}
对于单个平面波传播方程为:\[U=Ae^{2\pi i(\eta x+\xi y+\zeta z)}\]其中,
$A$为$\eta$、$\xi$、$z$的函数,代表的就是角谱,将这一方程带入亥姆霍兹方程可
以得到\[(\nabla^2+k^2)U=0\],展开得\[\frac{\dif^2A}{\dif
    z^2}+A(k^2-\eta^2-\xi^2)=0\]显然,方程的解为:
\begin{equation}
  \label{eq:2}
  A=Ce^{iz \sqrt{k^2-\eta^2-\xi^2}}
\end{equation}
$C$为和初值有关的常数,现在我们假设初始的光场角谱为$A_0$,则$C$可以简单
地取为$A_0$,$A_0$与只$\eta$、$\xi$有关,所以在传播$z$的距离后角谱变成
了
\begin{equation}
  \label{eq:3}
  A_z=A_0e^{iz \sqrt{k^2-\eta^2-\xi^2}}
\end{equation}
这就是角谱的传播规律。
\paragraph{}
由这也可以看出,当$k^2<\eta^2+\xi^2$时,开根会引入虚数,造成角谱传播过
程中的衰减,空间频率足够高的平面波将会消失掉。在这里做一个近似,将
$k^2<\eta^2+\xi^2$的地方角谱将近似为0,并定义:
\begin{equation}
  \label{eq:4}
  H(\eta,\xi)=\left \{
    \begin{array}{rcl}
      e^{iz \sqrt{k^2-\eta^2-\xi^2}}&&{k^2>\eta^2+\xi^2}\\
      0&&{}
    \end{array}
  \right
  .
\end{equation}
$H(\eta,\xi)$称为该系统的传递函数。

\subsection{泰伯效应}
\subsection{菲涅尔衍射}
\end{document}
%%% Local Variables:
%%% mode: latex
%%% TeX-master: t
%%% End:
